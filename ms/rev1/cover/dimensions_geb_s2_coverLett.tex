\documentclass[12pt]{article}
\usepackage[margin=1in]{geometry}
\geometry{letterpaper}
\usepackage{graphicx}
\usepackage{setspace}
\usepackage{amssymb}
\usepackage{amsmath}
\usepackage{epstopdf}

% for inline enumeration env
\usepackage{paralist}

%% to make enumeration more compact
% \usepackage{enumitem}
% \setlist{noitemsep}

\usepackage[compress,numbers]{natbib}


%%%%%%%%%%%%%% begin document %%%%%%%%%%%%%%
\begin{document}

\noindent
November 13, 2014
\vspace{2em}

\noindent
To the Editors of {\it Global Ecology \& Biogeography},

Please find enclosed our manuscript ``Community Assembly On Isolated
Islands:  Macroecology Meets Evolution''. This is a resubmission of the
manuscript submitted last June (2014) in response to an invitation
from Guida Santos, Bob Ricklefs, and Richard Field, to write an
article in the special issue ``New Directions in Island Biogeography.''
As before, the goal of this paper is to provide a framework in the use
of islands to understand how complex communities emerge from
ecological (population dynamics, dispersal, trophic interactions) and
evolutionary (genetic structuring, adaptation, speciation, extinction)
processes.  We synthesize data, mostly published, some new, through a
holistic approach and show that assembly by immigration gives way to
evolutionary processes, though at rates that differ according to the
trophic level of the organisms. Using ecological theory as a lens
through which to identify biologically interesting outliers, the
results play into our understanding of how changing community
characteristics should dictate metrics of species interaction
networks. Moreover, they provide insights into the evolution of the
networks they form.

Comments on the June submission were tremendously helpful in allowing
us to clarify, restructure, and streamline the manuscript. We have
reworked the entire manuscript to highlight the primary message of the
paper---which is to show how the Hawaiian Island chronosequence in
combination with novel ecological theory can be used to understand
both how communities develop over ecological--evolutionary time, and
the dynamic feedbacks involved in the assembly.  We do this by (1)
providing a framework highlighting the value of this approach and the
data and theory that are needed; and (2) using the preliminary data
that we have available as a demonstration of the potential of the
framework.

Sincerely,

Rosemary Gillespie
Corresponding author

\end{document}

