\documentclass[12pt]{article}
\usepackage[margin=1in]{geometry}
\geometry{letterpaper}
\usepackage{graphicx}
\usepackage{setspace}
\usepackage{amssymb}
\usepackage{amsmath}
\usepackage{epstopdf}

% for inline enumeration env
\usepackage{paralist}

%% to make enumeration more compact
% \usepackage{enumitem}
% \setlist{noitemsep}

\usepackage[compress,numbers]{natbib}


%%%%%%%%%%%%%% begin document %%%%%%%%%%%%%%
\begin{document}

\begin{center}
{\large \bf Response to reviewer comments}
\end{center}
\vspace{2em}

We thank Prof. Ricklefs and the two anonymous reviewers for their
detailed comments that have enormously improved the clarity and
quality of our manuscript.  Below we detail our response to their input.

\subsection*{Comments from handling editor, Prof. Robert Ricklefs}

\begin{itemize}
\item Part of the problem is that many of the statistical techniques,
  particularly with respect to network theory, are insufficiently
  explained for most readers. The structure of the sampling regime is
  not adequately described, and it would appear to be not strictly
  comparable between the islands.

\textit{\textbf{We have substantially re-written the methods and results
    sections to better detail statistical techniques and make more
    clear how focal sites were selected and plant-herbivore
    interaction data were compiled.}}
  % 
\item A major problem from my perspective is that the ``hypotheses''
  refer to qualitative expectations of patterns and are not closely
  linked to underlying mechanisms. Thus, the ``tests'' of these
  hypotheses do not discriminate between well--stated alternative
  mechanisms, but relate more to generalized expectations based
  loosely on ecological and evolutionary theory.  

  \textit{\textbf{We have re-framed the paper to focus first on
      discovering new patterns in the context of the Hawaiian
      chronosequence using ecological theory as a guide to identify
      potentially interesting biological outliers. We then use these
      patterns to generate new hypotheses about the feedback between
      ecological and evolutionary mechanisms in community assembly}}
  % 
\item[Line 34] it is not clear that HI is beyond equilibrium dynamics;
  perhaps the steady state is very low; if this has a different
  meaning, then it should be made more explicit.

  \textit{\textbf{This has been changed to make no claim about
      equilibrium}}
%
\item[Line 36] by ecological dynamics, do you mean long term
  environmental change caused by post-eruption succession? This should
  be explicit in the abstract.

  \textit{\textbf{We have made more clear that by ``ecological
      dynamics'' we are referring to those mechanisms such as biotic
      filtering and demography more often associated with ecological
      time scales.}}
%
\item[Line 52] presumably you are referring to species within clades
  of organisms at lower trophic levels; the trophic level itself
  doesn't develop divergence. The same comment applies to the next
  sentence.

\textit{\textbf{Changed accordingly}}
%
\item[Lines 142-3] predictive power doesn't come from patterns

\textit{\textbf{This phrasing has been removed}}
%
\item[Lines 169 ff.] these predictions appear to be ad hoc, and not based on
an underlying set of mechanisms. Why, for example, would primary
succession necessarily override the consequences of non-random
colonization from the mainland? What is it about the process of
succession that might produce this effect, and how would one test for
it in comparison with the outcome of alternatives? The basic problem
is to use patterns to distinguish between underlying processes, which
make unique predictions. Otherwise, this becomes an exercise in
describing pattern rather than using pattern to distinguish between
hypotheses (useful in its own right, of course).

\textit{\textbf{Further analysis and more detailed data are certainly
    needed to use pattern to distinguish process. In recognition of
    that fact we have refocused our paper to present patterns and use
    preliminary analyses to generate more meaningful hypotheses that
    can be tested with data that are fast accumulating about the biota
    of volcanic archipelagos.}}
%
\item[Lines 213 ff.] it might be good to state explicitly that you
  are going to limit your comparisons to ohia forests at mid
  elevation, if that is in fact how you are going to approach the
  problem.

  \textit{\textbf{We have made this more explicit in the methods
      section}}
%
\item[Lines 245 ff.] at what level is genetic variation relevant to
  the potential for divergence? Variation might be categorized within
  a locality, within a volcano, and between volcanoes on the same
  island; Fst values are mostly based on neutral variation, and
  perhaps are not so relevant to differentiation by selection. How
  will you use Fst values to estimate the potential for evolutionary
  differentiation and perhaps species diversification within islands?

\textit{\textbf{We have made more explicit that for evolution of novel
  variants to arise gene flow must be reduced; $F_{ST}$ captures the
  spatial structure that comes from reduced gene flow.}}
%
\item[Lines 251 ff.] it is not explicitly stated whether data exist
  on the distribution of herbivores across plant species within each
  site. How exactly were the bipartite networks constructed?

  \textit{\textbf{We have further clarified this in the methods
      section, making clear that site-specific data do not exist for
      plant host use so instead we make the reasonable assumption that
      if a known host plant is present at a site it will eventually be
      used and thus included in the network.}}
%
\item[Lines 269 ff.] this description will be understood only by
  individuals who work on network analysis.

  \textit{\textbf{We have further explained nestedness and modularity
      in the methods section for non-specialist readers}}
%
\item[Line 290 ff.] this hypothesis is more an assumption, in the
  sense that it is not based on an underlying process regarding how
  evolution should lead to specialization over time. The
  specialist-generalist trade-off does not appear to evolve in a
  single direction, and different levels of specialization might be
  optimal depending on all sorts of considerations. Thus, ``testing the
  hypothesis'' is really more like 'describing the pattern', since
  theory concerning the specialization-generalization spectrum is not
  taken up here. The same comment applies to the next paragraph, and
  really concerns how we understand the word ``hypothesis.''

  \textit{\textbf{This hypothesis has been removed}}
%
\item[Line 308] it strikes me as odd that the variation among sites
  within volcanoes is greater than the variation between volcanoes-or
  am I interpreting these values incorrectly? It also seems from
  Figure 1 that the distribution of sampling sites on the three
  islands is quite different, which would seem to confound space and
  age.

  \textit{\textbf{We agree that this result is unexpected. Rather than
      the more typical isolation by distance scenario such a
      partitioning of molecular variance could instead be consistent
      with, for example, a process of repeated colonization events
      between volcanoes leading to most molecular variance being
      present on many volcanoes while little variance is explained by
      unique variation between volcanoes. The primary difference in
      sampling between Maui and Big Island is that genetic data are
      only available from one volcano on Maui thus between volcano
      comparisons are not possible. }}
%
\item[Line 324] some of the Hawaii species apparently have divergence
  times older than the island, i.e., 1.15 Ma $>$ 0.5 Ma. How are we to
  interpret genetic variation within the species that are older than
  the island, as much of the genomic variation in the ancestral
  population is carried by a small number of colonists; or are the
  older age estimates the consequence of extinction or limiting
  sampling?

  \textit{\textbf{Bias of some sort is aways possible, or an incorrect
      calibration.  It could also be that alleles that get to the big
      island (even in one individual) are older than the big island
      (hopping down island chain), and if those alleles get separated
      in to different populations, and then diverge to species, the
      genetic variation observed between groups could be older than
      the habitat.}}
%
\item[Line 339] what exactly does the idea of maximum entropy
  represent, and what is its status as a null model. Since this is an
  important point, a little more explanation would be helpful to the
  uninitiated reader.

  \textit{\textbf{We have greatly expanded the conceptual motivation
      for maximum entropy, highlighting its interpretation in the
      context of ecology and clarifying that it serves to provide a
      yard stick against which to compare real assemblages. If those
      assemblages fail to meet the predictions of maximum entropy then
      there is evidence for the system to be out of statistical
      equilibrium and thus warrant further exploration of the
      mechanisms that could lead to that deviation.}}
%
\item[Lines 345 ff.] this is beginning to sound a little ad hoc, in
  terms of explanations for patterns.

\textit{\textbf{We have provided more motivation for analyses in the
    methods section}}
%
\item[Line 368] can you make a concluding sentence that describes
  these dynamics, or at least the change in patterns from youngest to
  oldest?

  \textit{\textbf{This phrasing no longer exists in the new submission}}
%
\item[Lines 391 ff.] this is an extremely interesting statement that
  would benefit from additional explanation.

  \textit{\textbf{The statement of interest comes from the work of
      Bennett \& O'Grady (2013). We have made this source more clear
      and contextualized it more within our own results.}}

\item We have also attended to the following more minor corrections: 
  \begin{itemize}
  \item[Lines 20-22] these superscript numbers do not appear in the
    author list.
  \item[Line 59] how about ``Our analyses suggest''
  \item[Line 94] italicize Anolis
  \item[Line 415] ``suggest''
  \item[Line 488] ``assembly''
  \item[Line 542] ``REFERENCES''
  \end{itemize}
\end{itemize}


\subsection*{Comments from Reviewer 1}

\begin{itemize}
\item The genetic data and network analyses don't seem to hang
  together to make a clear coherent story. Instead they seem merely
  suggestive that older communities are structured by evolutionary
  processes, and it is not always clear that other explanations for
  the patterns observed can be ruled out. Overall, the paper has a
  ``preliminary'' feel to it and I believe it would be much improved
  if the intentions specified in the ``Future research'' section of
  the paper were carried out.

  \textit{\textbf{Similarly to comments from Prof. Ricklefs, we have
      addressed this by re-focusing the paper to more explicitly
      address how the model system of Hawaiian arthropods can be used
      to address eco-evolutionary questions. We then present our
      analysis of mostly published data as a preliminary
      proof-of-concept.}}
%
\item There is quite a bit of literature around these days looking at
  ecological vs. evolutionary processes as drivers of community
  assembly. I suggest the authors look at the following to at least
  provide some deeper background to their thinking about this
  question.

  \textit{\textbf{ We have now included a discussion of the literature
      and how our study fits within the broader context. We have
      incorporated as much of this new literature as possible, though
      are constrained by citation limitations.}}
%
\item[Lines 169 and 178] More explanation is needed for the two
  hypotheses. For hypothesis 1, why should early assembly appear
  random?  Are there other circumstances that would produce random
  assembly? For hypothesis 2 more explanation is needed for why
  evolutionary processes should result in higher levels of
  specialization. I would think that this might depend upon the type
  of selection regime. Can species sorting through time also result in
  specialization?

  \textit{\textbf{These hypotheses no longer exist in the new
      submission.}}
%
\item I am not sufficiently knowledgeable to judge the molecular
  techniques or divergence dating analysis. However I note that mtDNA
  have been used in at least one case to look at divergence
  times. Coyne has found that if closely related species hybridize,
  then mtDNA cannot be used for this purpose because the mitochondria
  of one species can be completely captured by the other, thus
  obscuring divergence. The authors need to show that this is not a
  problem for their analysis.

  \textit{\textbf{It is possible that these species are older than we
      inferred from differences in their mtDNA (if similar DNA gets
      into 2 species it takes longer for the species to show DNA
      difference), but certainly the data show that some divergence
      between species is of the time scale noted, that is, quite
      recent. We now mention this issue (end of third paragraph of
      Discussion).}}
%
\item Generally, more explanation of network analysis and how it is
  used and what can be inferred from it is needed. Ditto for
  maximizing information entropy. The authors cite other papers that
  use these methods, but these methods should be explained clearly in
  the supplement to this paper so they can be easily evaluated.

  \textit{\textbf{We have expanded the methods and results section to
      better explain the origin and meaning of the results. We have
      also included content in the introduction to better motivate
      conceptually the use of maximum entropy in our analysis.}}
%
\item[Line 213] I would use the term ``replicate climatic and biotic
  templates'' rather than ``replicate communities'' here since the
  communities you are studying have different assembly and
  evolutionary histories.

  \textit{\textbf{This wording has been removed}}
%
\item[Line 228] and elsewhere. accession numbers not given

  \textit{\textbf{Accession numbers will be procured upon acceptance
      for publication}}
%
\item[Line 278] What aggregate properties were maintained in the null
  models and how might they affect you inferences. Some info is given
  in the supplement but more detail is needed.

  \textit{\textbf{We have further clarified the null models in the
      methods section}}
%
\item[Line 282] probabilistic and degree distribution models need more
  explanation, again in the supplement. Degree has a jargony
  character. It is used several places in the paper and it is not
  always clear what it means.

  \textit{\textbf{We have further clarified the meaning of the degree
      distribution, its derivation, use and interpretation, in the
      introduction and methods section}}
%
  \item[Line 322] I didn't see anywhere else in the paper a discussion
    of this phylogenetic data. Is it different from the genetic data?

    \textit{\textbf{We have clarified that these phylogenetic data are
        from relevant publications but not the genetic data we
        analyze.}}
%
\item[Lines 340-352] Patterns here are difficult to interpret (at
  least for me)

  \textit{\textbf{We have added further discussion of the generalized
      linear model results in the discussion section}}
\end{itemize}


\subsection*{Comments from Reviewer 2}

\begin{itemize}
\item I think the manuscript do a poor job in connecting the results
  reported with the vast literature on how ecological and evolutionary
  processes shape the geographical variation of species
  interactions. For example, it is safe to argue that the current
  theoretical framework to understand the evolutionary outcomes of
  species interactions is the Theory of the Geographic Mosaic of
  Coevolution introduced by John N. Thompson and developed by multiple
  research groups across the world. Accordingly, observational work,
  lab experiments and a variety of mathematical and computational
  models have been used to explore the role of species interactions on
  diversification. It is primal to a paper that aims to contribute to
  the integration of metacommunity ecology and evolution to do not
  ignore the current status of the literature and to actually show in
  which ways the paper provides news insights and contribute to the
  available theory (ies).

  \textit{\textbf{The Geographic Mosaic Theory constitutes an
      evolutionary hypothesis of selection mosaics, coevolutionary hot
      and cold spots, and trait remixing by dispersal; it aims to
      explain how coevolution reshapes interactions across different
      spatial and temporal scales, and analyzes the role of species
      interactions on diversification. Exploring the geographic mosaic
      of coevolution in the context of island chronologies would
      indeed present extremely interesting opportunities to further
      understand the micro-evolution of species
      interactions. Geographic mosaic theory has been developed
      largely in isolation from the network theoretic approach, a
      point acknowledged by Thompson himself in collaboration with
      Guimaraes and others (Guimaraes et al. 2006 Proc. R. Soc. B.)
      and thus integrating the two is a worthy task, but beyond the
      scope of our paper. Our approach is more in line with the
      network theoretic work, focusing on how the structure of
      interactions of a whole community changes over time and space,
      rather than on diversification of traits. However, the
      Geographic Mosaic theory, and in particular the role of trait
      remixing and selection mosaics, will be one of the key areas we
      will explore as we accumulate population genetic data across the
      chronosequence. We have included this point in the Future
      Research.}}
%
\item Some of the key predictions tested in this manuscript are
  related on how evolutionary processes should shape network
  structure. Unfortunately, these predictions are based on assumptions
  that do not hold. For example, we have now multiple theoretical
  evidence suggesting nestedness would be favored by natural selection
  (the work by Suweiss), differential extinction of species (the work
  by Fontaine and Thebault) and processes minimizing competition (the
  work by Bastolla). Therefore, to assume non-evolutionary processes
  would generate nestedness is not supported for our current
  theory. Moreover, to assume that evolution would necessarily favor
  specialization, and consequently, modularity is not supported by the
  evidence provided by both empirical and theoretical work on the
  evolution of species interactions in which a consumer can attack
  multiple prey species. Thus, the authors need to revisit carefully
  their assumptions and hypotheses to see if their results can still
  be presented as evidence of increased role of evolutionary processes
  in older islands or if alternative explanations are more likely to
  explain the observed pattern. In my opinion, and ignoring my
  concerns related to the data used (see below), I think that the
  results may provide evidence for particular outcomes of the
  evolutionary process in interactions among plants and herbivores.

  \textit{\textbf{In line with further comments from Reviewer 2 and
      from Prof. Ricklefs, we do not yet have sufficient data to use
      pattern to distinguish between multiple competing hypotheses
      about process. Thus instead of setting out to test hypotheses we
      use existing data and a combination of analytical techniques to
      generate a set of more meaningful hypotheses that will soon be
      testable with mounting ecological and evolutionary data from
      island systems. However, we have to disagree that theory
      strictly supports a deep time evolutionary cause of
      nestedness. Indeed the work of Suweis, claiming an abundance
      optimization criterion, is in conflict with the stability
      optimization arguments of Bastolla, Fontaine and Thebault. The
      work of Suweis is also consistent with previous work, which we
      cite, that nestedness is a likely outcome of random assembly as
      the crux of Suweis’s argument is that increased abundance,
      regardless of origin, increases nestedness. A primary goal of
      our paper is to provide a framework for explicit tests of these
      such ideas.}}
%
\item The use of maximum entropy theory in the manuscript is
  hermetic. The authors mention in multiple parts of the text the
  notion of the statistical steady state of ecological systems and how
  this steady state provides a theoretical benchmark. Nevertheless,
  there is no mention on how these expected distributions look
  like. For example, what is the degree distribution predicted by the
  maximum entropy theory? Even if Harte (2011) provides the expected
  distributions this missing information does not allow a reader not
  deeply familiarized maximum entropy theory to understand the results
  reports and the usefulness of maximum entropy theory for this work.

  \textit{\textbf{This is a very valid point, and one we hope has now
      been addressed: in the introduction and methods sections we have
      greatly expanded our description and conceptual motivation for
      using maximum entropy.}}
%
\item Because most of the available evidence for macroecological
  studies is correlative, it is usual to control for multiple
  confounding variables that may affect the reported patterns, leading
  to spurious correlations. Indeed, this careful exploration sets a
  control for inference on spatial and temporal trends in large-scale
  studies in ecology. I missed this kind controlled analysis in the
  current manuscript and without any kind of sensitivity analysis
  there is no enough evidence to support - in my opinion - the view
  that age is the best explanation for the observed patterns. I would
  like to know how the observed patterns are affect by island size or
  climate and habitat heterogeneity. To assume the islands are
  climatic similar may not be enough and there are ways of performing
  more rigorous analysis.

  \textit{\textbf{We agree that such statistical control is the core
      analytical strength of macroecology. Data do not currently exist
      to make such detailed analysis possible. Here again we have
      re-cast our paper, acknowledging this data limitation, to focus
      on using existing data to generate a richer set of hypotheses to
      be tested on similarly richer data.}}
%
\item I have two major concerns with the network analysis performed by
  the authors. First, I am afraid that the trends described are much
  less evident than stated by the authors. There is no linear or
  monotonic change in nestedness and modularity with island
  age. Indeed, there are three islands showing higher nestedness and
  the old one showing lower nestedness. Accordingly, there are two
  islands showing more modular networks and two areas showing less
  modular networks. Therefore, there is no strong evidence supporting
  the predictions that island age would favor modular and less nested
  networks. The authors need to review their interpretation of the
  results to avoid overemphasize the mentioned trends and, actually,
  to double-check if there is any trend. Second, it is impossible to
  verify in the current draft the quality of the data used to build up
  the ecological networks. The only information available is: ``We
  compiled plant-herbivore networks from published sources as
  described in the main text. Table 1 lists publications used in
  compiling these networks'' (Supplement). Nevertheless the Table 1 of
  the Supplement is not available for reviewing (``As part of the final
  submission we will make available our compiled list of Hemiptera (a
  typo here) and their plant hosts from published sources''). As a
  consequence, it is impossible to me, as a reviewer, to make any
  evaluation on how adequate is the dataset used to build up the
  networks and how comparable the networks are. Thus, the central
  information for this manuscript is missing and I am afraid I cannot
  just believe that the data are adequate for the analysis
  performed. Along the same lines, it is not clear to me if the
  interactions were actually observed in each island or just assumed
  because there is information in the literature that a given insect
  feeds on a given co-occurring plant elsewhere. If the latter case is
  true, all the analysis is invalidated as a test of local,
  evolutionary assembly of ecological networks since it will be based
  on the assumption that there is no local specialization.

  \textit{\textbf{In line with similar concerns from Prof. Ricklefs we
      have re-cast the manuscript as presenting a framework for
      integrating ecological and evolutionary perspectives on
      biodiversity using island chronosequecnes and ecological theory
      to do so. We have also made network construction and analysis
      more transparent in the methods. From this it should be clear
      that indeed interactions are derived from the spatial
      coincidence of herbivores and their potential host plants. This
      undeniably ignores the possibility of local specialization;
      however, local specialization is unlikely due to the already
      restricted (i.e. localized) ranges from which species
      descriptions are based and because the taxonomic level of
      specialization is typically that of genera.}}
%
\item[Abstract] the way network analysis is mentioned in the methods
  do not make clear why the authors performed this set of analysis.

  \textit{\textbf{We now make clear that these metrics summarize
      network structure.}}
%
\item[Line 70] To state that ``genomic and computational techniques
  have re-opened scientists' eyes to the rich and dynamic interplay
  between these processes'' (ecology and evolution) is unfair with the
  work of multiple research groups that have been using multiple
  approaches, from naturalistic observations to detailed lab
  experiments, to explore the interplay of ecology and evolution.

  \textit{\textbf{We have removed this language}}
%
\item[Line 84] I missed some discussion on the limitations of using
  islands as study case.

  \textit{\textbf{We have generally clarified our discussion of
      oceanic archipelagos as model systems. While the advantages of
      island systems are largely universal (e.g. their biotic
      simplification, discretization in space and time) their
      limitations are system specific and not unlike limitations
      possibly encountered in any mainland systems (habitat
      degradation, invasive species, etc.). Thus we do not address
      these simultaneously case-specific and yet non-unique
      limitations.}}
%
\item[Line 106] This is a quite limited definition of ecology and I
  would doubt that ecologists that are not community ecologists would
  agree.

  \textit{\textbf{We have removed reference to ``ecology'' and instead
      refer to ``ecological mechanisms'' such competition and neutral
      drift.}}
%
\item[Line 116] The use of network theory is much older than suggested
  in this sentence. See the work by Cohen, Margalef, Odum, Paine, May,
  and Pimm and the vast literature on food webs on the 70s and 80s.

  \textit{\textbf{While we acknowledge this earlier work, the more
      relevant application of network theory in ecology to our work
      comes more recently as indicated by our citations.}} 
%
\item[Line 133] This paragraph would be benefited for a synthesis of
  what we know about the integration of the mentioned processes.

  \textit{\textbf{We have substantially re-worked the introduction to
      now include more context.}}
%
\item[Line 282] Please provide the actual equations describing the
  probability of two species interact in each null model. This
  information should be available at least in the supplement.

  \textit{\textbf{We have provided a more detailed verbal description
      of these null models and full detailes are available in the
      works cited.}}
%
\item[Line 283] It is odd to call a species an ``island
  cosmopolitan.''

  \textit{\textbf{We retain this term as ``cosmopolitan'' refers to
      widely distributed species and here we wish to specifically
      refer to those species widely distributed across the islands,
      but not globally.}}
%
\item[Line 458] Specialization is not evidence for
  coevolution. Coevolution can favor generalization and specialization
  can emergence in the absence of coevolution.

  \textit{\textbf{Language conflating specialization and coevolution
      has been removed}}
%
\item[Lines 487 to 491] This part is speculative and based in a
  non-published paper that is not available to the audience yet. I
  suggest removing it.
%
\textit{\textbf{The section has been removed.}}
%
\item[Line 488] A typo: assembly.  

  \textit{\textbf{Corrected}}
\end{itemize}

\end{document}

